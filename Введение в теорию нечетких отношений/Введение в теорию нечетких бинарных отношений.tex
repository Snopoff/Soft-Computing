\documentclass[11pt,a4paper,oneside]{article}

%%% Работа с русским языком
\usepackage{cmap}					% поиск в PDF
\usepackage{mathtext} 				% русские буквы в фомулах
\usepackage[T2A]{fontenc}			% кодировка
\usepackage[utf8]{inputenc}			% кодировка исходного текста
\usepackage[english,russian]{babel}	% локализация и переносы

\usepackage{blindtext}

\usepackage{fancyhdr}

\usepackage{lipsum}
\usepackage{etoolbox}

% Code in Latex
\usepackage{listings}

%%% Дополнительная работа с математикой
\usepackage{amsmath,amsfonts,amssymb,amsthm,mathtools} % AMS
\usepackage{icomma} % "Умная" запятая: $0,2$ --- число, $0, 2$ --- перечисление

\title{Практическая работа по теме "Введение в теорию нечетких бинарных отношений"}
\date{2020}
\author{Снопов П.М.}

\newenvironment{problem}{
	\medskip
	\begin{problem-internal}
	}{
	\end{problem-internal}
}

\newenvironment{solution}{
	\begin{proof}[Решение]
		\vspace{-8px}
		\setlength{\parskip}{4px}
		\setlength{\parindent}{0px}
	}{
	\end{proof}
}

\newtheorem{problem-internal}{}

\begin{document}
	\maketitle
	
	\begin{problem}
		Приведите пример функции принадлежности нечеткого отношения $R =$ { \bf значительно меньше, чем в $\mathbb{N} \times \mathbb{N}$}. Ограничиваясь первыми десятью натуральными числами, постройте матрицу этого отношения.
	\end{problem}

	\begin{solution}
		Нечеткое отношение $R$ определено своей функцией принадлежности. Тогда зададим функцию следующим образом: 
		\[\mu_R: \mathbb{N} \times \mathbb{N} \to [0,1]: (x,y) \mapsto \begin{cases}
		\dfrac{1}{1 + exp(x-y)} & x < y \\
		0 & \text{иначе}
		\end{cases}
		\]
		Тогда матрица этого отношения будет иметь вид:
		\begin{table}[h!]
			\begin{tabular}{|l|l|l|l|l|l|l|l|l|l|l|}
				\hline
				$R$& 1 & 2    & 3    & 4    & 5    & 6    & 7    & 8    & 9    & 10   \\ \hline
				1  & 0 & 0.73 & 0.88 & 0.95 & 0.98 & 0.99 & 1    & 1    & 1    & 1    \\ \hline
				2  & 0 & 0    & 0.73 & 0.88 & 0.95 & 0.98 & 0.99 & 1    & 1    & 1    \\ \hline
				3  & 0 & 0    & 0    & 0.73 & 0.88 & 0.95 & 0.98 & 0.99 & 1    & 1    \\ \hline
				4  & 0 & 0    & 0    & 0    & 0.73 & 0.88 & 0.95 & 0.98 & 0.99 & 1    \\ \hline
				5  & 0 & 0    & 0    & 0    & 0    & 0.73 & 0.88 & 0.95 & 0.98 & 0.99 \\ \hline
				6  & 0 & 0    & 0    & 0    & 0    & 0    & 0.73 & 0.88 & 0.95 & 0.98 \\ \hline
				7  & 0 & 0    & 0    & 0    & 0    & 0    & 0    & 0.73 & 0.88 & 0.95 \\ \hline
				8  & 0 & 0    & 0    & 0    & 0    & 0    & 0    & 0    & 0.73 & 0.88 \\ \hline
				9  & 0 & 0    & 0    & 0    & 0    & 0    & 0    & 0    & 0    & 0.73 \\ \hline
				10 & 0 & 0    & 0    & 0    & 0    & 0    & 0    & 0    & 0    & 0    \\ \hline
			\end{tabular}
		\end{table}
	
	\end{solution}

	\newpage
	
	\begin{problem}
		Постройте первую, вторую и глобальную проекции для нечеткого отношения, заданного матрицей: 
		\begin{table}[!htbp]
			\begin{tabular}{|l|l|l|l|l|l|}
				\hline
				$R_1$ & a   & b   & c   & d   & e   \\ \hline
				a     & 0.3 & 1   & 0.2 & 0.1 & 0.5 \\ \hline
				b     & 0.9 & 0.2 & 0   & 0.5 & 0.1 \\ \hline
				c     & 0.8 & 0.1 & 0.8 & 0.9 & 0.9 \\ \hline
				d     & 0.9 & 0.5 & 1   & 0.9 & 0.1 \\ \hline
				e     & 0.5 & 0   & 0.7 & 0.7 & 0.8 \\ \hline
			\end{tabular}
		\end{table}
	\end{problem}	

	\begin{solution}
		Так как первая проекция $\pi_1$ имеет вид $\pi_1(x) = \sup\limits_{y \in Y} \mu_R(x,y)$, тогда:
		\begin{table}[!htbp]
			\begin{tabular}{|l|l|l|l|l|l|}
				\hline
				$\pi_1$ & a & b   & c   & d & e   \\ \hline
				& 1 & 0.9 & 0.9 & 1 & 0.8 \\ \hline
			\end{tabular}
		\end{table}
		\newline
		В свою очередь вторая проекция $\pi_1$ имеет вид $\pi_2(y) = \sup\limits_{x \in X} \mu_R(x,y) $, то:
		\begin{table}[!htbp]
			\begin{tabular}{|l|l|l|l|l|l|}
				\hline
				$\pi_2$ & a   & b & c & d   & e   \\ \hline
				& 0.9 & 1 & 1 & 0.9 & 0.9 \\ \hline
			\end{tabular}
		\end{table}
		\newline
		Глобальная проекция $h(R)$ имеет вид: $h(R) = \sup\limits_x \pi_1(x) = \sup\limits_y \pi_2(y) $, значит $h(R) = 1$. Таким образом, $R$ -- нормальное отношение.
	\end{solution}
		
	\newpage
	
	\begin{problem}
		Заданы отношения $R_1$, $R_2$, $R_3$: 
		\begin{table}[!hbtp]
			\begin{tabular}{|l|l|l|l|l|}
				\hline
				$R_1$ & $y_1$ & $y_2$ & $y_3$ & $y_4$ \\ \hline
				$x_1$ & 0     & 0.1   & 0     & 0.4   \\ \hline
				$x_2$ & 0.5   & 1     & 0     & 0.7   \\ \hline
				$x_3$ & 0.8   & 0.9   & 0.9   & 1     \\ \hline
			\end{tabular}
		\end{table}
		\begin{table}[!hbtp]
			\begin{tabular}{|l|l|l|l|l|}
				\hline
				$R_2$ & $y_1$ & $y_2$ & $y_3$ & $y_4$ \\ \hline
				$x_1$ & 0.1     & 0   & 0.2     & 0.5   \\ \hline
				$x_2$ & 0   & 1     & 0.1     & 1   \\ \hline
				$x_3$ & 0.9   & 0.4   & 0.7   & 0     \\ \hline
			\end{tabular}
		\end{table}
		\begin{table}[!hbtp]
			\begin{tabular}{|l|l|l|l|l|}
				\hline
				$R_3$ & $y_1$ & $y_2$ & $y_3$ & $y_4$ \\ \hline
				$x_1$ & 0.5     & 0   & 0.2     & 0   \\ \hline
				$x_2$ & 0   & 1     & 0.1     & 0.2   \\ \hline
				$x_3$ & 0.9   & 0.4   & 0   & 1     \\ \hline
			\end{tabular}
		\end{table}
		\newline
		Определите: а) $R_1 \cap R_2$; б) $R_1 \cup R_3$; в) $ R_1 \oplus R_2 $; г) $ (\overline{R_1} \cap \overline{R_2}) \oplus R_3 $
	\end{problem}
	
	\begin{solution}
		а) $R_1 \cap R_2: \mu_{R_1 \cap R_2}(x,y) = \min\{\mu_{R_1}(x,y), \mu_{R_2}(x,y)\}$, тогда:
		\begin{table}[!hbtp]
			\begin{tabular}{|l|l|l|l|l|}
				\hline
				$R_1 \cap R_2$ & $y_1$ & $y_2$ & $y_3$ & $y_4$ \\ \hline
				$x_1$ & 0     & 0   & 0     & 0.4   \\ \hline
				$x_2$ & 0   & 1     & 0     & 0.7   \\ \hline
				$x_3$ & 0.8   & 0.4   & 0.7   & 0     \\ \hline
			\end{tabular}
		\end{table}
	
		б) $R_1 \cup R_3: \mu_{R_1 \cup R_3}(x,y) = \max\{\mu_{R_1}(x,y), \mu_{R_3}(x,y)\}$, тогда:
		\begin{table}[!hbtp]
			\begin{tabular}{|l|l|l|l|l|}
				\hline
				$R_1 \cup R_2$ & $y_1$ & $y_2$ & $y_3$ & $y_4$ \\ \hline
				$x_1$ & 0.5     & 0.1   & 0.2     & 0.4   \\ \hline
				$x_2$ & 0.5   & 1     & 0.1     & 0.7   \\ \hline
				$x_3$ & 0.9   & 0.9   & 0.9 & 1     \\ \hline
			\end{tabular}
		\end{table}
	
		в) $ R_1 \oplus R_2 : \mu_{R_1 \oplus R_2}(x,y) = \mu_{R_1}(x,y) + \mu_{R_2}(x,y) - \mu_{R_1}(x,y)\mu_{R_2}(x,y) $, тогда:
		\begin{table}[!hbtp]
			\begin{tabular}{|l|l|l|l|l|}
				\hline
				$R_1 \oplus R_2$ & $y_1$ & $y_2$ & $y_3$ & $y_4$ \\ \hline
				$x_1$ & 0.1     & 0.1   & 0.2     & 0.7    \\ \hline
				$x_2$ & 0.5   & 1     & 0.1     & 1  \\ \hline
				$x_3$ & 0.98   & 0.94   & 0.97 & 1     \\ \hline
			\end{tabular}
		\end{table}
	
		\newpage
		г) $(\overline{R_1} \cap \overline{R_2}) \oplus R_3 = \overline{R_1 \cup R_2} \oplus R_3$, тогда:
		\begin{table}[!hbtp]
			\begin{tabular}{|l|l|l|l|l|}
				\hline
				$(\overline{R_1} \cap \overline{R_2}) \oplus R_3$ & $y_1$ & $y_2$ & $y_3$ & $y_4$ \\ \hline
				$x_1$ & 0.95     & 0.9   & 0.84     & 0.5   \\ \hline
				$x_2$ & 0.5   & 1     & 0.91     & 0.2   \\ \hline
				$x_3$ & 0.91   & 0.46   & 0.1   & 1     \\ \hline
			\end{tabular}
		\end{table}
	
	\end{solution}
	\begin{problem}
		Определите для отношений $R_1$ и $R_2$, заданных матрицами, следующие композиции: $(\max-\min)$, $(\min-\max)$, $(\max-\cdot)$, $(\min-\oplus)$:
		\begin{table}[!hbtp]
			\begin{tabular}{|l|l|l|l|l|}
				\hline
				$R_1$ & $y_1$ & $y_2$ & $y_3$ & $y_4$ \\ \hline
				$x_1$ & 0.3   & 0     & 0.7   & 0.3   \\ \hline
				$x_2$ & 0     & 1     & 0.2   & 0     \\ \hline
			\end{tabular}
		\end{table}
		\begin{table}[!hbtp]
			\begin{tabular}{|l|l|l|l|}
				\hline
				$R_2$ & $z_1$ & $z_2$ & $z_3$ \\ \hline
				$y_1$ & 1     & 0     & 1     \\ \hline
				$y_2$ & 0     & 0.5   & 0.4   \\ \hline
				$y_3$ & 0.7   & 0.9   & 0.6   \\ \hline
				$y_4$ & 0     & 0     & 0     \\ \hline
			\end{tabular}
		\end{table}
		\newline
		Можно ли говорить о включении полученных отношений?
	\end{problem}
	\begin{solution}
		1) $(\max-\min)-\text{композиция } R_1 \circ R_2: \mu_{R_1 \circ R_2}(x,z) = \max\limits_{y \in Y} \min\{ \mu_{R_1}(x,y) , \mu_{R_2}(y,z) \}$, тогда имеем следующую матрицу:
		\begin{table}[!hbtp]
			\begin{tabular}{|c|c|c|c|}
				\hline
				$R_1 \circ R_2$ & $z_1$ & $z_2$ & $z_3$ \\ \hline
				$x_1$ & 0.7     & 0.7     & 0.6     \\ \hline
				$x_2$ & 0.2     & 0.5   & 0.4   \\ \hline
			\end{tabular}
		\end{table}
		\newline
		2) $(\min-\max)-\text{композиция } R_1 \ast R_2: \mu_{R_1 \ast R_2}(x,z) = \min\limits_{y \in Y} \max\{ \mu_{R_1}(x,y) , \mu_{R_2}(y,z) \}$
		\begin{table}[!hbtp]
			\begin{tabular}{|c|c|c|c|}
				\hline
				$R_1 \ast R_2$ & $z_1$ & $z_2$ & $z_3$ \\ \hline
				$x_1$ & 0     & 0.3     & 0.3     \\ \hline
				$x_2$ & 0     & 0   & 0   \\ \hline
			\end{tabular}
		\end{table}
		\newpage
		3) $(\max-\cdot)-\text{композиция } R_1 \cdot R_2: \mu_{R_1 \cdot R_2}(x,z) = \max\limits_{y \in Y} \{ \mu_{R_1}(x,y) \cdot \mu_{R_2}(y,z) \}$
		\begin{table}[!hbtp]
			\begin{tabular}{|c|c|c|c|}
				\hline
				$R_1 \cdot R_2$ & $z_1$ & $z_2$ & $z_3$ \\ \hline
				$x_1$ & 0.49  & 0.63& 0.42     \\ \hline
				$x_2$ & 0.14     & 0.5   & 0.4   \\ \hline
			\end{tabular}
		\end{table}
		\newline
		4) $(\min-\oplus)-\text{композиция } R_1 \oplus R_2: \mu_{R_1 \oplus R_2}(x,z) = \min\limits_{y \in Y} \{ \mu_{R_1}(x,y) \oplus \mu_{R_2}(y,z) \}$
		\begin{table}[!hbtp]
			\begin{tabular}{|c|c|c|c|}
				\hline
				$R_1 \oplus R_2$ & $z_1$ & $z_2$ & $z_3$ \\ \hline
				$x_1$ & 0     & 0.3     & 0.3     \\ \hline
				$x_2$ & 0     & 0   & 0   \\ \hline
			\end{tabular}
		\end{table}
		\newline
		Легко увидеть, что $ R_1 \oplus R_2 = R_1 \ast R_2 \subset  R_1 \cdot R_2 \subset R_1 \circ R_2 $
	\end{solution}
\end{document}