\documentclass[11pt,a4paper,oneside]{article}

%%% Работа с русским языком
\usepackage{cmap}					% поиск в PDF
\usepackage{mathtext} 				% русские буквы в фомулах
\usepackage[T2A]{fontenc}			% кодировка
\usepackage[utf8]{inputenc}			% кодировка исходного текста
\usepackage[english,russian]{babel}	% локализация и переносы

\usepackage{blindtext}

\usepackage{fancyhdr}

\usepackage{lipsum}
\usepackage{etoolbox}

% Code in Latex
\usepackage{listings}

%%% Дополнительная работа с математикой
\usepackage{amsmath,amsfonts,amssymb,amsthm,mathtools} % AMS
\usepackage{icomma} % "Умная" запятая: $0,2$ --- число, $0, 2$ --- перечисление

\title{Свойства и типы нечетких бинарных отношений}
\date{2020}
\author{Снопов П.М.}

\newenvironment{problem}{
	\medskip
	\begin{problem-internal}
	}{
	\end{problem-internal}
}

\newenvironment{solution}{
	\begin{proof}[Решение]
		\vspace{-8px}
		\setlength{\parskip}{4px}
		\setlength{\parindent}{0px}
	}{
	\end{proof}
}

\newtheorem{problem-internal}{}
\newcommand{\+}{$\boldsymbol{+}$}

\begin{document}
	\maketitle
	
	%{\it Замечание:} В данном задании все дальнейшие вычисления были сделаны с помощью Python. %Прилагаю файл, где все вычисления описаны: 
	
	
	
	\begin{problem}
		Определите, какие из перечисленных ниже нечетких бинарных отношений симметричны,
		антисимметричны, совеCршенно антисимметричны, рефлексивны, транзитивны:
		\begin{table}[!htbp]
			\centering
			\begin{tabular}{|c|c|c|c|c|c|}
				\hline
				$R_1$ & a & b & c   & d   & e   \\ \hline
				a     & 0 & 1 & 1   & 1   & 1   \\ \hline
				b     & 0 & 0 & 0.9 & 0.7 & 0.3 \\ \hline
				c     & 0 & 0 & 0   & 0.7 & 0.3 \\ \hline
				d     & 0 & 0 & 0   & 0   & 0.3 \\ \hline
				e     & 0 & 0 & 0   & 0   & 0   \\ \hline
			\end{tabular}
			\quad
			\centering
			\begin{tabular}{|c|c|c|c|c|c|}
				\hline
				$R_2$ & a   & b   & c   & d   & e   \\ \hline
				a     & 0   & 0.3 & 1   & 0   & 0.5 \\ \hline
				b     & 0.3 & 0.2 & 0   & 0.8 & 0.1 \\ \hline
				c     & 1   & 0   & 0   & 0.2 & 1   \\ \hline
				d     & 0   & 0.8 & 0.2 & 1   & 0.4 \\ \hline
				e     & 0.5 & 0.1 & 1   & 0.4 & 0.4 \\ \hline
			\end{tabular}
		\end{table}
		\begin{table}[!htbp]
			\centering
			\begin{tabular}{|c|c|c|c|c|c|}
				\hline
				$R_3$ & a & b   & c   & d & e   \\ \hline
				a     & 1 & 0.5 & 0.5 & 0 & 0.7 \\ \hline
				b     & 0 & 1   & 0.7 & 0 & 0   \\ \hline
				c     & 0 & 1   & 1   & 0 & 0   \\ \hline
				d     & 0 & 0.3 & 0.3 & 1 & 0   \\ \hline
				e     & 1 & 0.5 & 0.5 & 0 & 1   \\ \hline
			\end{tabular}
			\quad
			\centering
			\begin{tabular}{|c|c|c|c|c|c|}
				\hline
				$R_4$ & a   & b & c   & d   & e   \\ \hline
				a     & 0   & 0 & 0.3 & 0.2 & 0   \\ \hline
				b     & 0.6 & 1 & 0.8 & 1   & 0.2 \\ \hline
				c     & 0.2 & 0 & 1   & 0.8 & 0.3 \\ \hline
				d     & 0   & 0 & 0   & 1   & 0   \\ \hline
				e     & 1   & 0 & 0.2 & 0.6 & 0   \\ \hline
			\end{tabular}
		\end{table}
	\end{problem}
	
	\begin{solution}
		Построим таблицу, в которой укажем, какие отношения обладают вышеобозначенными свойствами. Обозначим за $С$ -- симметричность, $АС$ -- антисимметричность, $САС$ -- совершенно антисимметричность, $Р$ -- рефлексивность и $Т$ -- транзитивность:
		\begin{table}[!htbp]
			\centering
			\begin{tabular}{|c|c|c|c|c|}
				\hline
						& $R_1$ & $R_2$ & $R_3$ & $R_4$ \\ \hline
				$С$   	&       &   \+  &       &       \\ \hline
				$АС$  	&   \+  &       &  \+   &  \+   \\ \hline
				$САС$ 	&   \+  &       &       &       \\ \hline
				$Р$   	&       &       &   \+  &       \\ \hline
				$Т$   	&   \+  &       &   \+  &       \\ \hline
			\end{tabular}
		\end{table}
	
	\end{solution}
	\begin{problem}
		Найдите максминное транзитивное замыкание любых двух отношений из 1 задания.
	\end{problem}
	\begin{solution}
		Найдем транзитивные замыкания первого и четвертого отношений:
		\begin{table}[!hbtp]
			\centering
			\begin{tabular}{|c|c|c|c|c|c|}
				\hline
				$\hat{R}_1$ & a & b & c   & d   & e   \\ \hline
				a     & 0 & 1 & 1   & 1   & 1   \\ \hline
				b     & 0 & 0 & 0.9 & 0.7 & 0.3 \\ \hline
				c     & 0 & 0 & 0   & 0.7 & 0.3 \\ \hline
				d     & 0 & 0 & 0   & 0   & 0.3 \\ \hline
				e     & 0 & 0 & 0   & 0   & 0   \\ \hline
			\end{tabular}
			\quad
			\begin{tabular}{|c|c|c|c|c|c|}
				\hline
				$\hat{R}_4$ & a   & b & c   & d   & e   \\ \hline
				a     & 0.3 & 0.0 & 0.3 & 0.3 & 0.3 \\ \hline
				b     & 0.6 & 1.0 & 0.8 & 1.0 & 0.3 \\ \hline
				c     & 0.3 & 0.0 & 1.0 & 0.8 & 0.3 \\ \hline
				d     & 0.0 & 0.0 & 0.0 & 1.0 & 0.0 \\ \hline
				e     & 1.0 & 0.0 & 0.3 & 0.6 & 0.3 \\ \hline
			\end{tabular}
		\end{table}
		\newline
		Так как $R_1$ транзитивно, то $\hat{R}_1 = R_1$
	\end{solution}

	\begin{problem}
		Докажите, что нечеткое отношение $R$, представленное ниже, является предпорядком:
		\begin{table}[!hbtp]
			\centering
			\begin{tabular}{|c|c|c|c|c|c|c|}
				\hline
				$R$ & a & b   & c   & d & e   & f   \\ \hline
				a   & 1 & 0.7 & 0.2 & 0 & 0.8 & 1   \\ \hline
				b   & 0 & 1   & 0.2 & 0 & 0   & 0   \\ \hline
				c   & 0 & 0.5 & 1   & 0 & 0   & 0   \\ \hline
				d   & 0 & 0.1 & 0.1 & 1 & 0.1 & 0.1 \\ \hline
				e   & 0 & 0   & 0   & 0 & 1   & 0.8 \\ \hline
				f   & 0 & 0   & 0   & 0 & 0.8 & 1   \\ \hline
			\end{tabular}
		\end{table}
	\end{problem}
	\begin{solution}
		$R$ -- предпорядок, если $R$ рефлексивно и транзитивно. Очевидно, $R$ рефлексивно. Также $R$ транзитивно:
		\begin{table}[!hbtp]
			\centering
			\begin{tabular}{|c|c|c|c|c|c|c|}
				\hline
				$R \circ R$ & a  & b   & c   & d   & e   & f   \\ \hline
				 a  		&1.0 & 0.7 & 0.2 & 0.0 & 0.8 & 1.0 \\  \hline
				 b  		&0.0 & 1.0 & 0.2 & 0.0 & 0.0 & 0.0 \\  \hline
				 c  		&0.0 & 0.5 & 1.0 & 0.0 & 0.0 & 0.0 \\  \hline
				 d  		&0.0 & 0.1 & 0.1 & 1.0 & 0.1 & 0.1 \\ \hline
				 e  		&0.0 & 0.0 & 0.0 & 0.0 & 1.0 & 0.8 \\ \hline
				 f  		&0.0 & 0.0 & 0.0 & 0.0 & 0.8 & 1.0 \\ \hline
			\end{tabular}
		\end{table}
		
		Значит $R$ -- предпорядок
	\end{solution}
	\newpage
	\begin{problem}
		 Является ли следующее отношение подобием?
		 \begin{table}[!hbtp]
		 	\centering
		 	\begin{tabular}{|c|c|c|c|c|c|c|}
		 		\hline
		 		$R$ & a   & b   & c   & d   & e   & f   \\ \hline
		 		a   & 1   & 0.1 & 0.1 & 0   & 0   & 0.5 \\ \hline
		 		b   & 0.1 & 1   & 0.6 & 0   & 0   & 0.1 \\ \hline
		 		c   & 0.1 & 0.6 & 1   & 0   & 0   & 0.1 \\ \hline
		 		d   & 0   & 0   & 0   & 1   & 0.3 & 0   \\ \hline
		 		e   & 0   & 0   & 0   & 0.3 & 1   & 0   \\ \hline
		 		f   & 0.5 & 0.1 & 0.1 & 0   & 0   & 1   \\ \hline
		 	\end{tabular}
		 \end{table}
	\end{problem}
	\begin{solution}
		Отношение $R$ является отношением подобия(отношением эквивалентности), если оно рефлексивно, симметрично и транзитивно. Очевидно, $R$ рефлексивно и симметрично. Проверим транзитивность: 
		\begin{table}[!hbtp]
			\centering
			\begin{tabular}{|c|c|c|c|c|c|c|}
				\hline
				$R \circ R$ & a  &    b& c	 & d   & e	 & f \\ \hline
				a			&1.0 & 0.1 & 0.1 & 0.0 & 0.0 & 0.5 \\ \hline
				b			&0.1 & 1.0 & 0.6 & 0.0 & 0.0 & 0.1 \\ \hline
				c			&0.1 & 0.6 & 1.0 & 0.0 & 0.0 & 0.1 \\ \hline
				d			&0.0 & 0.0 & 0.0 & 1.0 & 0.3 & 0.0 \\ \hline
				e			&0.0 & 0.0 & 0.0 & 0.3 & 1.0 & 0.0 \\ \hline
				f			&0.5 & 0.1 & 0.1 & 0.0 & 0.0 & 1.0 \\ \hline
			\end{tabular}
		\end{table}
	
		Значит, $R$ -- отношение эквивалентности.
	\end{solution}
\end{document}
